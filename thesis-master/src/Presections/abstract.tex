\mbox{}
%\clearpage
\setstretch{1.3}  % Reset the line-spacing to 1.3 for body text (if it has changed)

% The Abstract Page
\addtotoc{Abstract}  % Add the "Abstract" page entry to the Contents
\abstract{

Diese Arbeit ist eine Analyse, des in \cite{Doettling2021} vorgestellten Protokolls.
In dem Paper wurde ein verbessertes Protokoll vorgestellt, das es beliebig vielen Parteien ermöglicht, die Schnittmenge ihrer Eingabemengen sicher zu berechnen. Ein großer Teil dieser Berechnung ist die Ermittlung der Größe der Schnittmenge. Die effiziente Lösung dieses Problems ist der Fokus des Papers.\\
Um das Protokoll zu analysieren, habe ich die relevanten Teil-Protokolle wie im Paper beschrieben in Java implementiert, und die Funktionalitäten der weniger interessanten, auch schon in anderen Papern beschriebenen, Protokolle abgebildet. Zusätzlich habe ich für die grundlegenden Funktionalitäten auch eine additiv homomorphe Verschlüsselung in Java implementiert und diese so erweitert, dass sie auch Multiplikation ermöglicht.\\
Die Analyse zeigt, dass das vorgestellte Protokoll wie beschrieben funktioniert und auch effizient ist.\\
Dadurch, dass nun gezeigt ist, dass das Protokoll nützlich ist, kann nun Forschung im Bereich der secure computation weiter auf dem Protokoll aufbauen, mit der Gewissheit, dass es keine grundlegenden Fehler enthält. 

\addtocontents{toc}{\vspace{1em}}  % Add a gap in the Contents, for aesthetics

\clearpage  % Abstract ended, start a new page
