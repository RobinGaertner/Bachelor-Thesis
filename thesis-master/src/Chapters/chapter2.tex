\chapter{Grundlagen}


\section{Homomorphe Verschlüsselungen}
Homomorphe Verschlüsselung ist eine Form der Verschlüsselung, die bestimmte Berechnungen auf verschlüsselten Daten erlaubt, um dann ein verschlüsseltes Ergebnis zurückzugeben. Wenn dieses dann wieder entschlüsselt wird, ist das Ergebnis das gleiche, wie wenn diese Arbeitsschritte auf dem unverschlüsselten Anfangswert ausgeführt worden wären. \cite{Yi2014} \\
In dieser Arbeit wird im speziellen additiv homomorphe Verschlüsselung benutzt. Diese  erlaubt das Addieren von zwei verschlüsselten Zahlen und das Multiplizieren einer verschlüsselten mit einer unverschlüsselten Zahl.\\
Diese Funktionalitäten sind für das Protokoll wichtig, denn in jedem Unterprotokoll wird mit verschlüsselten Daten gerechnet. Und nur so können wir trotz einer sicheren Verschlüsselung mit den Daten rechnen, ohne den Inhalt der Verschlüsselung zu kennen.

\section{Secure Computation}
Secure computation ist ein Teil der Kryptographie. Das Ziel der secure computation ist es, den Teilnehmern des Protokolls zu erlauben eine Funktion ihrer geheimen Eingabewerte zu berechnen, ohne dass etwas anderes als das gewollte Ergebnis der Berechnung öffentlich wird. \cite{cryptoeprint:2020:300} Die grundlegendste Form der secure computation erlaubt nur die Teilnahme von Zwei unabhängigen Parteien, viele Protokolle erlauben jedoch sichere Berechnungen zwischen mehreren teilnehmenden Parteien. Die wichtigsten Eigenschaften der Protokolle sind Privacy und Correctness. Privacy bedeutet, dass niemand Informationen über die Eingaben der anderen erhält (außer das ist Teil des Rückgabewerts). Correctness bedeutet, dass jeder Teilnehmer sicher sein kann, dass das Ergebnis korrekt ist. \cite{cryptoeprint:2020:300} Es gibt mehrere mathematische Grundlagen, auf denen secure computation aufbauen kann. Eine Grundlage ist beispielsweise Shamir´s Secret Sharing. Secure computation hat viele mögliche Einsatzbereiche. Bei Einigen steht die Privacy und bei Anderen die Correctness im Vordergrund \cite{cryptoeprint:2020:300}. In vielen Bereichen können dann Protokolle, bei denen man vorher den anderen Parteien vertrauen musste, durch Protokolle ersetzt werden, die kein Vertrauen in die anderen Parteien erfordern, und das ist hat viele Vorteile, auch wenn dafür die Kosten der Berechnung steigen.