\chapter{Grundlagen}


\section{Homomorphe Verschlüsselungen}
Homomorphe Verschlüsselung ist eine Form der Verschlüsselung, die bestimmte Berechnungen auf verschlüsselten Daten erlaubt, um dann ein verschlüsseltes Ergebnis zurückzugeben. Wenn dieses dann wieder entschlüsselt wird, ist das Ergebnis das gleiche, wie wenn diese Arbeitsschritte auf dem unverschlüsselten Anfangswert ausgeführt worden wären. \cite{Yi2014} \\
In dieser Arbeit wird im speziellen additiv homomorphe Verschlüsselung benutzt. Diese  erlaubt das Addieren von zwei verschlüsselten Zahlen und das Multiplizieren einer verschlüsselten mit einer unverschlüsselten Zahl.\\
Diese Funktionalitäten sind für das Protokoll wichtig, denn in jedem Unterprotokoll wird mit verschlüsselten Daten gerechnet. Und nur so können wir trotz einer sicheren Verschlüsselung mit den Daten rechnen, ohne den Inhalt der Verschlüsselung zu kennen.
