\chapter{Grundlagen}

In diesem Kapitel werden einige Grundlagen erläutert, die zum besseren Verständnis der späteren Kapitel beitragen können.

\section{Verschlüsselungen mit mehreren Parteien}
Bei Verschlüsselungen mit mehr als zwei Parteien kann ein Threshold-Verschlüsselungssystem sinnvoll sein. Bei diesen Verschlüsselungssystemen werden Teile des Private Keys unter allen Parteien aufgeteilt. Um diesen Private Key dann zu nutzen, müssen genügend (mehr als ein gegebener threshold) Parteien zusammenarbeiten.\\
Das kann hilfreich sein, damit ein Angreifer die Kontrolle über viele Parteien (mehr als der threshold) haben muss, um den Private key nutzen zu können. Andererseits kann der Private Key jedoch auch genutzt werden, ohne dass immer die Mitarbeit aller beteiligter Parteien nötig ist.\\

\section{Homomorphe Verschlüsselungen}
Homomorphe Verschlüsselung \cite{Yi2014} ist eine Form der Verschlüsselung, die bestimmte Berechnungen auf verschlüsselten Daten erlaubt, um dann ein verschlüsseltes Ergebnis zurückzugeben. Wenn dieses dann wieder entschlüsselt wird, ist das Ergebnis das gleiche, wie wenn diese Arbeitsschritte auf dem unverschlüsselten Anfangswert ausgeführt worden wären.\\
In dieser Arbeit im Speziellen wird additiv homomorphe Verschlüsselung benutzt. Diese  erlaubt das Addieren von zwei verschlüsselten Zahlen und das Multiplizieren einer verschlüsselten mit einer unverschlüsselten Zahl.\\
Diese Funktionalitäten sind für das Protokoll wichtig, denn in jedem Unterprotokoll wird mit verschlüsselten Daten gerechnet. Und nur so können wir trotz einer sicheren Verschlüsselung mit den Daten rechnen, ohne den Inhalt der Verschlüsselung zu kennen.\\
Die Multiplikation von zwei verschlüsselten Zahlen ermöglicht viele neue Funktionalitäten. Wenn Multiplikationen uneingeschränkt möglich sind, nennt man die Verschlüsselung auch voll homomorphe Verschlüsselung.\\

\section{Secure Computation}
Secure Computation ist ein Teil der Kryptographie. Das Ziel der Secure Computation ist es, den Teilnehmern des Protokolls zu erlauben eine Funktion ihrer geheimen Eingabewerte zu berechnen, ohne dass etwas anderes als das gewollte Ergebnis der Berechnung öffentlich wird. \cite{cryptoeprint:2020:300} Die grundlegendste Form der secure computation erlaubt nur die Teilnahme von zwei unabhängigen Parteien. Viele Protokolle erlauben jedoch sichere Berechnungen zwischen mehreren teilnehmenden Parteien. Die wichtigsten Eigenschaften der Protokolle sind Privacy und Correctness. Privacy bedeutet, dass niemand Informationen über die Eingaben der anderen erhält (außer das ist Teil des Rückgabewerts). Correctness bedeutet, dass jeder Teilnehmer sicher sein kann, dass das Ergebnis korrekt ist. \cite{cryptoeprint:2020:300} Es gibt mehrere mathematische Grundlagen, auf denen secure computation aufbauen kann. Eine Grundlage ist beispielsweise Shamirs Secret Sharing. Secure computation hat viele mögliche Einsatzbereiche \cite{cryptoeprint:2020:300}. In vielen Bereichen können dann Protokolle, bei denen man vorher den anderen Parteien vertrauen musste, durch Protokolle ersetzt werden, die kein Vertrauen in die anderen Parteien erfordern, und das hat viele Vorteile, auch wenn dafür die Kosten der Berechnung steigen.

\section{Threshold PSI}
PSI steht für \glqq Private Set Intersection\grqq{} und beschreibt die Berechnung der Schnittmenge der Eingabemengen. Dabei soll jedoch nur das Ergebnis veröffentlicht werden, wir befinden uns also im Bereich der Secure Computation.
Dabei kann es noch einige spezielle Anforderungen geben. Eine dieser Anforderungen ist, dass die Schnittmenge nur ausgegeben wird, wenn sie größer ist, als ein gegebener Wert. Eine andere Anforderung wäre, dass die Schnittmenge von mehreren Eingabemengen von mehreren Parteien berechnet werden soll. 
Diese Anforderung kann auch mit der vorherigen Anforderung kombiniert werden, sodass wie im Fall dieser Arbeit von \glqq Muliparty Threshold PSI\grqq gesprochen wird.

