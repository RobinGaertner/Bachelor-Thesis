\chapter{Analyse}


\section{Die Effizienz von MPCT}
Das Protokoll MPCT, das die Größe der Schnittmenge der Eingabemengen sicher berechnet, ist bei meinen Tests auch bei insgesamt 100*99, also 9.900 Eingaben sehr schnell gewesen und hat auch bei rund 10.000 Eingaben schon nach wenigen Sekunden an secDT übergeben.\\
\begin{lstlisting}[caption = Ausschnitt von Rückgabe von Test MPCTTestBig. Dauer von MPCT ohne SDT]
MPCT start: 2021-07-28 16:52:52.62
MPCT end: 2021-07-28 16:52:57.376
\end{lstlisting}
Das Protokoll MPCT benötigt also nur wenige Berechnungen, um die Eingabemengen so vorzubereiten, dass das nächste Teil-Protokoll secDT das Ergebnis berechnen kann. Die Berechnungsdauer von MPCT hängt also zu einem großen Teil von der Berechnungsdauer von secDT ab. 

\section{Die Effizienz von secDT}
Das Protokoll secDT, das berechnet, ob der Grad der Eingabepolynome kleiner als ein gegebener Wert ist, hat in meiner Analyse länger gebraucht, als MPCT ohne secDT. Das liegt jedoch daran, dass secDT auf mehreren anderen Teilprotokollen basiert. Diese anderen Teilprotokolle, wie OLS sind teilweise sehr rechenintensiv. Vor allem, da sie nicht wie im Paper beschrieben implementiert werden konnten. Wenn man aber die Berechnungszeit der anderen Teilprotokolle abzieht kann man einen besseren Überblick erhalten, wie effizient seccDT ist.

\begin{lstlisting}[caption = Ausschnitt von Rückgabe von Test MPCTTestBig. Dauer von SDT]
SDT start: 2021-07-28 17:05:15.126
getRank start: 2021-07-28 17:05:15.407
getRank end: 2021-07-28 17:08:21.118
getRank start: 2021-07-28 17:08:21.125
getRank end: 2021-07-28 17:11:41.791
getRank start: 2021-07-28 17:11:42.232
getRank end: 2021-07-28 17:14:51.385
getRank start: 2021-07-28 17:14:51.39
getRank end: 2021-07-28 17:18:07.677
OLS start: 2021-07-28 17:18:08.135
OLS end: 2021-07-28 17:21:27.384
OLS start: 2021-07-28 17:21:27.385
OLS end: 2021-07-28 17:24:47.028
SDT end: 2021-07-28 17:24:50.787
\end{lstlisting}

19:31 min hat die die Berechnung von SDT insgesamt gedauert. Die Berechnung der vier Aufrufe von secRank hat rund 12:51 min gedauert und die Berechnung der zwei Aufrufe von OLS rund 6:49 min. Wenn man also die Berechnungszeit der beiden ineffizient implementierten Teilprotokolle abzieht, benötigt auch SDT nur rund 5 Sekunden um das Ergebnis der 42 verschlüsselten Eingaben zu berechnen. Diese 42 verschlüsselten Eingaben entsprechen ebenfalls den fast 10.000 Eingaben, die MPCT erhalten hat.


\section{Die Berechnungen in beiden Protokollen}
Im Folgenden werden die Berechnungen in secDT und MPCT zusammen betrachtet.
Die Berechnungen in anderen Teilprotokollen werden nicht analysiert, weil sie nicht wie im Paper \cite{Doettling2021} beschrieben implementiert wurden.

   \begin{table}[!h]
     \centering
     \begin{tabular}{ccccccc}
       \textbf{Testname} & \textbf{Parteien} & \textbf{Listenlänge} & erlaubte Abweichungen & Rank & decrypt & encrypt \\
       MPCTTest5Parties & 5 & 5 & 2 & 4 & 4 & 116\\
       MPCTTest10Numbers & 2 & 10 & 2 & 4 & 4 & 68\\
       MPCTTest & 2 & 5 & 3 & 4 & 4 & 88\\
       MPCTTestBigTreshold & 2 & 10 & 9 & 4 & 4 & 208\\
       MPCTTestBig &40 & 99 & 10 & 4 & 4 & 2052\\
     \end{tabular}

     \caption{Tabelle der Testergebnisse}
     \label{tbl:results}
     % Verweis im Text mittels \ref{tbl:beispieltabelle}

   \end{table}
   
\section{Analyseergebnisse}
