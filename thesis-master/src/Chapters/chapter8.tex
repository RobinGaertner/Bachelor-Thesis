\chapter{Fazit}

Das Ziel war es, besser abschätzen zu können, wie schnell und effizient das MPCT Protokoll durch die Neuerungen von Branco et al. \cite{Doettling2021} wird.\\
Die dort vorgestellten Teilprotokolle MPCT und SDT funktionieren problemlos, solange das Modulo der Verschlüsselung groß genug ist.\\
Die Analyse der Teilprotokolle MPCT und secDT hat ergeben, dass die Größe der Eingabemengen die geringsten Auswirkungen aller Eingabeparameter hat. Die Laufzeit hängt nun also nicht mehr vor Allem von der Größe der Eingabemengen ab, sondern  von der Anzahl der \glqq erlaubten Abweichungen\grqq{} oder dem \glqq threshold\grqq, wie er im Paper genannt wird, und der Anzahl der teilnehmenden Parteien. Die Anzahl der Teilnehmer hat dabei größere Auswirkungen, als der threshold.\\
Die Berechnungszeit des kompletten Protokolls hängt von der Implementierung und wie in der Analyse gezeigt zu einem großen Teil von der Effizienz der verwendeten externen Protokolle ab. Da die externen Protokolle als \glqq effizient\grqq{} beschrieben werden, könnte die Geschwindigkeit des komplett sicher implementierten Protokolls der Geschwindigkeit dieser Test-Implementierung ähneln. Dabei muss jedoch bedacht werden, dass viele der Berechnungen ja von den verschiedenen Parteien gleichzeitig berechnet werden können. Für einige Testzeiten siehe im Anhang unter \ref{tbl:Times}\\
Die Implementierung der additiv homomorphen Damgard-Jurik Verschlüsselung und des Teilprotokolls secMult, das auch verschlüsselte Multiplikationen ermöglicht, funktionieren ebenfalls. Dieses Teilprotokoll wird für die korrekte Funktionalität von SDT benötigt.\\
Damit kann je nach Aufgabe und Anwendungsbereich sehr viel effizienter als in früheren Protokollen festgestellt werden, ob die Schnittmenge der Eingabemengen größer ist, als ein gegebener Wert. Das kann dann mit anderen Protokollen kombiniert werden, wie beispielsweise dem von Gosh und Simkin \cite{Ghosh2019} entworfenen Protokoll, um noch komplexere Protokolle effizienter zu machen. \\
Leider konnte ich durch die Zeitbegrenzung dieser Arbeit keine komplette, sichere Implementierung des ganzen vorgestellten Protokolls liefern, da zu viele der Teil-Protokolle auf die Implementierungen anderer Paper zurückgreifen, die es zu diesem Zeitpunkt noch nicht gibt. Das Protokoll ist so nicht direkt in Anwendungen nutzbar, die auf der Datensicherheit der Eingabemengen bestehen.\\
Das Ziel dieser Arbeit ist also erreicht, auch wenn bis zu einer sicheren Implementierung des kompletten Protokolls noch einige Arbeit nötig ist.